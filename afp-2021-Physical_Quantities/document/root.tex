\documentclass[11pt,a4paper]{book}
\usepackage{isabelle,isabellesym}
\usepackage{graphicx}
\graphicspath {{figures/}}

% further packages required for unusual symbols (see also
% isabellesym.sty), use only when needed


\usepackage{latexsym}
\usepackage{amssymb}
  %for \<leadsto>, \<box>, \<diamond>, \<sqsupset>, \<mho>, \<Join>,
  %\<lhd>, \<lesssim>, \<greatersim>, \<lessapprox>, \<greaterapprox>,
  %\<triangleq>, \<yen>, \<lozenge>

%\usepackage[greek,english]{babel}
  %option greek for \<euro>
  %option english (default language) for \<guillemotleft>, \<guillemotright>

%\usepackage[latin1]{inputenc}
  %for \<onesuperior>, \<onequarter>, \<twosuperior>, \<onehalf>,
  %\<threesuperior>, \<threequarters>, \<degree>

%\usepackage[only,bigsqcap]{stmaryrd}
  %for \<Sqinter>

%\usepackage{eufrak}
  %for \<AA> ... \<ZZ>, \<aa> ... \<zz> (also included in amssymb)

%\usepackage{textcomp}
  %for \<cent>, \<currency>

% this should be the last package used
\usepackage{pdfsetup}

% urls in roman style, theory text in math-similar italics
\urlstyle{rm}
\isabellestyle{it}
\newcommand{\HOL}[1]{\verb{HOL}}
\newcommand{\eg}[1]{e.g.}
\renewcommand{\isasymdegree}{XXX}
\newcommand{\acs}[1]{\textsc{#1}} 

\begin{document}

\title{A Sound Type System for Physical \\ Quantities, Units, and Measurements}
\author{Simon Foster \and Burkhart Wolff}

\maketitle

\chapter*{Abstract}
The present Isabelle theory builds a formal model for both the \emph{International System of Quantities} (ISQ) and the
\emph{International System of Units} (SI), which are both fundamental for physics and
engineering~\cite{bipm-jcgm:2012:VIM}. Both the ISQ and the SI are deeply integrated into Isabelle's type
system. Quantities are parameterised by \emph{dimension types}, which correspond to base vectors, and thus only
quantities of the same dimension can be equated. Since the underlying ``algebra of quantities''
from~\cite{bipm-jcgm:2012:VIM} induces congruences on quantity and SI types, specific tactic support is developed
to capture these. Our construction is validated by a test-set of known equivalences between both quantities and SI
units. Moreover, the presented theory can be used for type-safe conversions between the SI system and others, like the
British Imperial System (BIS).

\tableofcontents

% sane default for proof documents
\parindent 0pt\parskip 0.5ex


\chapter{ISQ and SI: An Introduction}

Modern Physics is based on the concept of quantifiable properties of physical phenomena such 
as mass, length, time, current, etc. These phenomena, called \emph{quantities}, are linked via an 
\emph{algebra of quantities} to derived concepts such as speed, force, and energy. The latter 
allows for a \emph{dimensional analysis} of physical equations, which had already been the 
backbone of Newtonian Physics. In parallel, physicians developed their own research field called 
``metrology'' defined as a scientific study of the \emph{measurement} of physical quantities.

The relevant international standard for quantities and measurements is distributed by the \emph{Bureau International des
  Poids et des Mesures} (BIPM), which also provides the \emph{Vocabulaire International de M\`etrologie}
(VIM)~\cite{bipm-jcgm:2012:VIM}.  The VIM actually defines two systems: the \emph{International System of Quantities}
(ISQ) and the \emph{International System of Units} (SI, abbreviated from the French Syst\`eme international
(d’unit\'es)). The latter is also documented in the \emph{SI Brochure}~\cite{SI-Brochure}, a standard that is updated
periodically, most recently in 2019. Finally, the VIM defines concrete reference measurement procedures as well as a
terminology for measurement errors.

Conceived as a refinement of the ISQ, the SI comprises a coherent system of units of measurement built on seven base
units, which are the metre, kilogram, second, ampere, kelvin, mole, candela, and a set of twenty prefixes to the unit
names and unit symbols, such as milli- and kilo-, that may be used when specifying multiples and fractions of the
units. The system also specifies names for 22 derived units, such as lumen and watt, for other common physical
quantities. While there is still nowadays a wealth of different measuring systems such as the \emph{British Imperial
  System} (BIS) and the \emph{United States Customary System} (USC), the SI is more or less the de-facto reference
behind all these systems.

The present Isabelle theory builds a formal model for both the ISQ and the SI, together with a deep integration into
Isabelle's type system~\cite{nipkow.ea:isabelle:2002}. Quantities and units are represented in a way that they have a
\emph{quantity type} as well as a \emph{unit type} based on its base vectors and their magnitudes. Since the algebra of
quantities induces congruences on quantity and SI types, specific tactic support has been developed to capture these.
Our construction is validated by a test-set of known equivalences between both quantities and SI units.  Moreover, the
presented theory can be used for type-safe conversions between the SI system and others, like the British Imperial
System (BIS).

% We would like to stress that it is not only our objective to provide a sound type-system for
% ISQ and SI; rather, our semantic construction produces an integration of quantities and SI units 
% \emph{as types} inside the Hindley-Milner style type system of
%  Isabelle/HOL\cite{nipkow.ea:isabelle:2002}. The objective of our construction is to
% reflect the types of the magnitudes as well as their dimensions in order to allow type-safe 
% calculations on SI units and their conversion to other measuring systems.

% The International System of Units (SI, abbreviated from the French
% Système International (d'unités)) is the modern form of the metric
% system and is the most widely used system of measurement. It comprises
% a coherent system of units of measurement built on seven base units,
% which are the second, metre, kilogram, ampere, kelvin, mole, candela,
% and a set of twenty prefixes to the unit names and unit symbols that
% may be used when specifying multiples and fractions of the units.
% The system also specifies names for 22 derived units, such as lumen and
% watt, for other common physical quantities.
% 
% (cited from \url{https://en.wikipedia.org/wiki/International_System_of_Units}).

In the following we describe the overall theory architecture in more detail.
Our ISQ model provides the following fundamental concepts:
%
\begin{enumerate}%
\item \emph{dimensions} represented by a type \isa{(int, 'd::enum) dimvec} , i.e. a \isa{'d}-indexed
      vector space of integers representing the exponents of the dimension vector. 
      \isa{'d} is constrained to be a dimension type later.


\item \emph{quantities} represented by type \isa{('a, 'd::enum) Quantity}, which are constructed as 
      a  vector space and a magnitude type \isa{'a}. 

\item{quantity calculus} consisting of \emph{quantity equations} allowing to infer that 
      $LT^{-1}T^{-1}M = MLT^{-2} = F$ 
      (the left-hand-side equals mass times acceleration which is equal to force). 

\item a kind of equivalence relation $\cong_{Q}$ on quantities, permitting to relate
      quantities of different dimension types.


\item \emph{base quantities} for
      \emph{length}, \emph{mass}, \emph{time}, \emph{electric current},
      \emph{temperature}, \emph{amount of substance}, and \emph{luminous intensity}, 
      serving as concrete instance of the vector instances, and for syntax
      a set of the symbols  \isa{L}, \isa{M}, \isa{T}, \isa{I},  
      \isa{{\isasymTheta}}, \isa{N}, \isa{J} corresponding to the above mentioned
      base vectors.

\item \emph{(Abstract) Measurement Systems} represented by type 
      \isa{('a, 'd::enum, 's::unit\_system) Measurement\_System}, which are a refinement
      of quantities. The refinement is modelled by a polymorphic record extensions; as a 
      consequence, Measurement Systems inherit the algebraic properties of quantities.
 
\item \emph{derived dimensions} such as \emph{volume} $\isa{L}^3$ or energy 
      $\isa{M}\isa{L}^2\isa{T}^{-2}$ corresponding to \emph{derived quantities}.

\end{enumerate}

Then, through a fresh type-constructor \isa{SI}, the abstract measurement systems are instantiated to the SI system ---
the \emph{British Imperial System} (BIS) is constructed analogously.  Technically, \isa{SI} is a tag-type that
represents the fact that the magnitude of a quantity is actually a quantifiable entity in the sense of the SI system. In
other words, this means that the magnitude $1$ in quantity \isa{1[L]} actually refers to one metre intended to be
measured according to the SI standard. At this point, it becomes impossible, for example, to add to one foot, in the
sense of the BIS, to one metre in the SI without creating a type-inconsistency.

The theory of the SI is created by specialising the \isa{Measurement\_System}-type with the 
SI-tag-type and adding new infrastructure. The SI theory provides the following fundamental 
concepts:
\begin{enumerate}%
\item measuring units and types corresponding to the ISQ base quantities sich
      as \emph{metre}, \emph{kilogram}, \emph{second}, \emph{ampere}, \emph{kelvin}, \emph{mole} and
      \emph{candela} (together with procedures how to measure a metre, for example, which are
      defined in accompanying standards);
\item a standardised set of symbols for units such as $m$, $kg$, $s$, $A$, $K$, $mol$, and $cd$;
\item a standardised set of symbols of SI prefixes for multiples of SI units, such as 
      $giga$ ($=10^9$), $kilo$ ($=10^3$), $milli$ ($=10^-3$), etc.; and a set of
\item \emph{unit equations} and conversion equations such as $J = kg\,m^2/s^2$ or $1 km/h = 1/3.6\,m/s$.
\end{enumerate}

As a result, it is possible to express ``4500.0 kilogram times metre per second squared'' which has the type
\isa{{\isasymreal}\ {\isacharbrackleft}M\ \isactrlsup {\isachardot}\ L\ \isactrlsup {\isachardot}\ T\isactrlsup
  {\isacharminus}\isactrlsup {\isadigit{3}} \isactrlsup {\isachardot}\, SI{\isacharbrackright}}.  This type means that
the magnitude $4500$ of the dimension $M \cdot L \cdot T^{- 3}$ is a quantity intended to be measured in the SI-system,
which means that it actually represents a force measured in Newtons.
% For short, the above expression gets thy type $(\isasymreal)newton$.  
In the example, the \emph{magnitude} type of the measurement unit is the real numbers ($\isasymreal$).  In general,
however, magnitude types can be arbitrary types from the HOL library, so for example integer numbers (\isa{int}),
integer numbers representable by 32 bits (\isa{int32}), IEEE-754 floating-point numbers (\isa{float}), or, a vector in
the three-dimensional space \isa{\isasymreal}$^3$. Thus, our type-system allows to capture both conceptual entities in
physics as well as implementation issues in concrete physical calculations on a computer.

As mentioned before, it is a main objective of this work to support the quantity calculus of ISQ and the resulting
equations on derived SI entities (cf. \cite{SI-Brochure}), both from a type checking as well as a proof-checking
perspective. Our design objectives are not easily reconciled, however, and so some substantial theory engineering is
required. On the one hand, we want a deep integration of dimensions and units into the Isabelle type system. On the
other, we need to do normal-form calculations on types, so that, for example, the units $m$ and $ms^{-1}s$ can be
equated.

Isabelle's type system follows the Curry-style paradigm, which rules out the possibility of direct calculations on
type-terms (in contrast to Coq-like systems). However, our semantic interpretation of ISQ and SI allows for the
foundation of the heterogeneous equivalence relation $\cong_{Q}$ in semantic terms. This means that we can relate
quantities with syntactically different dimension types, yet with same dimension semantics. This paves the way for
derived rules that do computations of terms, which represent type computations indirectly. This principle is the basis
for the tactic support, which allows for the dimensional type checking of key definitions of the SI system. Some
examples are given below.

\begin{isamarkuptext}%
\isa{\isacommand{theorem}\ metre{\isacharunderscore}definition{\isacharcolon} \newline \ {\isachardoublequoteopen} 
{\isadigit{1}}\ {\isacharasterisk}\isactrlsub Q\ metre\ {\isasymcong}\isactrlsub Q  \ {\isacharparenleft}\isactrlbold c\ \isactrlbold {\isacharslash}\ {\isacharparenleft}{\isadigit{2}}{\isadigit{9}}{\isadigit{9}}{\isadigit{7}}{\isadigit{9}}{\isadigit{2}}{\isadigit{4}}{\isadigit{5}}{\isadigit{8}}\ {\isacharasterisk}\isactrlsub Q\ {\isasymone}{\isacharparenright}{\isacharparenright}\isactrlbold {\isasymcdot}second{\isachardoublequoteclose}\ {\isachardoublequoteopen}
\newline \isacommand{by}\ si{\isacharunderscore}calc\ \ \newline \newline
\isacommand{theorem}\ kilogram{\isacharunderscore}definition{\isacharcolon} \newline \ {\isachardoublequoteopen}{\isadigit{1}}\ {\isacharasterisk}\isactrlsub Q\ kilogram\ {\isasymcong}\isactrlsub Q\ {\isacharparenleft}\isactrlbold h\ \isactrlbold {\isacharslash}\ {\isacharparenleft}{\isadigit{6}}{\isachardot}{\isadigit{6}}{\isadigit{2}}{\isadigit{6}}{\isadigit{0}}{\isadigit{7}}{\isadigit{0}}{\isadigit{1}}{\isadigit{5}}\ {\isasymcdot}\ {\isadigit{1}}{\isacharslash}{\isacharparenleft}{\isadigit{1}}{\isadigit{0}}{\isacharcircum}{\isadigit{3}}{\isadigit{4}}{\isacharparenright}\ {\isacharasterisk}\isactrlsub Q\ {\isasymone}{\isacharparenright}{\isacharparenright}\isactrlbold {\isasymcdot}metre\isactrlsup {\isacharminus}\isactrlsup {\isasymtwo}\isactrlbold {\isasymcdot}second{\isachardoublequoteclose}\ \newline \isacommand{by}\ si{\isacharunderscore}calc\ \ \ }%
\end{isamarkuptext}\isamarkuptrue%

These equations are both adapted from the SI Brochure, and give the concrete definitions for the metre and kilogram in
terms of the physical constants \textbf{c} (speed of light) and \textbf{h} (Planck constant). They are both proved
using the tactic \textit{si-calc}.

This work has drawn inspiration from some previous formalisations of the ISQ and SI, notably Hayes and Mahoney's
formalisation in Z~\cite{HayesBrendan95} and Aragon's algebraic structure for physical
quantities~\cite{Aragon2004-SI}. To the best of our knowledge, our mechanisation represents the most comprehensive
account of ISQ and SI in a theory prover.


% \subsubsection{Previous Attempts.} The work of \cite{HayesBrendan95} represents to our knowledge a
% first attempt to formalize SI units in Z, thus a similar language of HOL. While our typing
% representation is more rigourous due to the use of type-classes, this works lacks any attempt
% to support formal and automated deduction on Si unit equivalences.
%
% MORE TO COME.

\chapter{Preliminaries}

\input{session}

% optional bibliography
\bibliographystyle{abbrv}
\bibliography{root}

\end{document}

%%% Local Variables:
%%% mode: latex
%%% TeX-master: t
%%% End:
