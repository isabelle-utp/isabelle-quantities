\documentclass[11pt,a4paper]{book}
\usepackage{isabelle,isabellesym}
\usepackage{graphicx}
\graphicspath {{figures/}}

% further packages required for unusual symbols (see also
% isabellesym.sty), use only when needed


\usepackage{latexsym}
\usepackage{amssymb}
  %for \<leadsto>, \<box>, \<diamond>, \<sqsupset>, \<mho>, \<Join>,
  %\<lhd>, \<lesssim>, \<greatersim>, \<lessapprox>, \<greaterapprox>,
  %\<triangleq>, \<yen>, \<lozenge>

%\usepackage[greek,english]{babel}
  %option greek for \<euro>
  %option english (default language) for \<guillemotleft>, \<guillemotright>

%\usepackage[latin1]{inputenc}
  %for \<onesuperior>, \<onequarter>, \<twosuperior>, \<onehalf>,
  %\<threesuperior>, \<threequarters>, \<degree>

%\usepackage[only,bigsqcap]{stmaryrd}
  %for \<Sqinter>

%\usepackage{eufrak}
  %for \<AA> ... \<ZZ>, \<aa> ... \<zz> (also included in amssymb)

%\usepackage{textcomp}
  %for \<cent>, \<currency>

% this should be the last package used
\usepackage{pdfsetup}

% urls in roman style, theory text in math-similar italics
\urlstyle{rm}
\isabellestyle{it}
\newcommand{\HOL}[1]{\verb{HOL}}
\newcommand{\eg}[1]{e.g.}
\renewcommand{\isasymdegree}{XXX}
\newcommand{\acs}[1]{\textsc{#1}} 

\begin{document}

\title{A Sound Type-System for Physical Quantities and Measurements}
\author{Simon Foster \and Burkhart Wolff}
\maketitle

\textbf{ Abstract } 

The present Isabelle theory builds a formal model for both the \emph{International System of Quantities}
(ISQ) and the \emph{International System of Units} (SI), which are both fundamental for physics
and engineering~\cite{bipm-jcgm:2012:VIM}. Both the ISQ and the SI were deeply integrated into 
Isabelle's type system, i.e. quantities and units were represented in a way that 
they have a  \emph{quantity type} as well as \emph{unit type} based on its base vectors and their 
magnitudes. Since the underlying "algebra of quantities" from ~\cite{bipm-jcgm:2012:VIM} induces 
congruences on  quantity and SI types, specific tactic support has been developed to capture these.
Our construction is validated by a test-set of known equivalences between both quantities and SI units.
Moreover, the presented theory can be used for type-safe conversions between the SI system and
others, like the British Imperial System (BIS).

\tableofcontents

% sane default for proof documents
\parindent 0pt\parskip 0.5ex


\chapter{ISQ and SI: An Introduction}

Modern Physics is based on the concept of quantifiable properties of physical phenomena such 
as mass, length, time, etc. These phenomena, called \emph{quantities}, are linked via an 
\emph{algebra of quantities} to derived concepts such as speed, force, and energy. The latter 
allows for a \emph{dimensional analysis} of physical equations, which had already been the 
backbone of Newtonian Physics. In parallel, physicians developped an own research field called 
"metrology" defined as scientific study of the \emph{measurement} of physical quantities.


The relevant international standard for quantities and measurements is distributed by the
\emph{Bureau International des Poids et des Mesures} (BIPM), which also develops and distributes
the \emph{Vocabulaire International de M\`etrologie} (VIM) \cite{bipm-jcgm:2012:VIM}.
The VIM actually defines two systems: the \emph{International System of Quantities}
(ISQ) and the \emph{International System of Units} (SI, abbreviated from the French  Syst\`eme 
international (d’unit\'es)). Finally, the VIM defines concrete reference measurement procedures 
as well as a terminology for measurement errors. 

Conceived as a refinement of the ISQ, the SI comprises a coherent system of units of measurement built
 on seven base units, which are the second, metre, kilogram, ampere, kelvin, mole, candela, and a 
set of twenty prefixes to the unit names and unit symbols that may be used when specifying multiples 
and fractions of the units. The system also specifies names for 22 derived units, such as lumen and 
watt, for other common physical quantities. While there is still nowadays a wealth of different 
measuring systems such as the \emph{British Imperial System} (BIS) and the \emph{United States 
customary system} (USC), the SI became more or less the de-facto reference behind all these systems.

The present Isabelle theory builds a formal model for both the ISQ and the SI, together with a 
deep integration into Isabelle's type system\cite{nipkow.ea:isabelle:2002}. Quantities and units 
were represented in a way that they have a  \emph{quantity type} as well as \emph{unit type} based 
on its base vectors and their magnitudes. Since the algebra of quantities induces congruences on 
quantity and SI types, specific tactic support has been developed to capture these.
Our construction is validated by a test-set of known equivalences between both quantities and SI units.
Moreover, the presented theory can be used for type-safe conversions between the SI system and
others, like the British Imperial System (BIS).

% We would like to stress that it is not only our objective to provide a sound type-system for
% ISQ and SI; rather, our semantic construction produces an integration of quantities and SI units 
% \emph{as types} inside the Hindley-Milner style type system of
%  Isabelle/HOL\cite{nipkow.ea:isabelle:2002}. The objective of our construction is to
% reflect the types of the magnitudes as well as their dimensions in order to allow type-safe 
% calculations on SI units and their conversion to other measuring systems.

% The International System of Units (SI, abbreviated from the French
% Système International (d'unités)) is the modern form of the metric
% system and is the most widely used system of measurement. It comprises
% a coherent system of units of measurement built on seven base units,
% which are the second, metre, kilogram, ampere, kelvin, mole, candela,
% and a set of twenty prefixes to the unit names and unit symbols that
% may be used when specifying multiples and fractions of the units.
% The system also specifies names for 22 derived units, such as lumen and
% watt, for other common physical quantities.
% 
% (cited from \url{https://en.wikipedia.org/wiki/International_System_of_Units}).

In the following we describe the overall theory architecture in more detail.
Our ISQ model provides the following fundamental concepts:
%
\begin{enumerate}%
\item \emph{dimensions} represented by a type \isa{(int, 'd::enum) DimScheme} , i.e. an \isa{'d}-indexed
      vectorspace of integers representing the exponents of the dimension vector. 
      \isa{'d} is constrained to be a dimension type later.


\item \emph{quantities} represented by type \isa{('a, 'd::enum) Quantity}, which are constructed as 
      a  vectorspace and a magnitude type \isa{'a}. 

\item{quantity calculus} consisting of \emph{quantity equations} allowing to infer that 
      $\isa{L}\isa{T}^{-1}\isa{T}^{-1}\isa{M} = \isa{M}\isa{L}\isa{T}^{-2} = \isa{F} $ 
      (the left-hand-side equals mass times acceleration which is equal to force). 

\item a kind of equivalence relation $\_\cong_{Q}\_$ on quantities, permitting to relate
      quantities of different dimension type.


\item \emph{base vectors} for
      \emph{time}, \emph{length}, \emph{mass}, \emph{electric current},
      \emph{temperature}, \emph{amount of substance}, and \emph{luminous intensity}, 
      serving as concrete instance of the vector instances, and as syntax a
      a set of the symbols  \isa{T}, \isa{L}, \isa{M}, \isa{I},  
      \isa{{\isasymTheta}}, \isa{N}, \isa{J}  corresponding to the above mentioned base 
      base vectors.

\item \emph{(Abstract) Measurement Systems} represented by type 
      \isa{('a, 'd::enum, 's::unit\_system) Measurement\_System}, which are a refinement
      of quantities. The refinement is modeled by a polymorphic record extensions; as a 
      consequence, Measurement Systems inherit the algebraic properties of quantities.
 

\item \emph{derived dimensions} such as \emph{volume} $\isa{L}^3$ or energy 
      $\isa{M}\isa{L}^2\isa{T}^{-2}$ corresponding to \emph{derived quantities}.

\end{enumerate}

Via a fresh type-constructor \isa{SI}, the abstract measurement systems were instantiated
to the SI system --- the \emph{British Imperial System} (BIS) is constructed analogously.
Technically, \isa{SI} is a tag-type that represents the fact that the magnitude of a quantity is
actually a quantifiable entity in the sense of the SI system. In other words, this means that the 
magnitude $1$ in quantity \isa{1[L]} actually refers to one meter intended to be measured 
according to the SI standard. At this point, it becomes impossible, for example, to add to one 
foot in the sense of the BIS one meter without creating a type-inconsistency.

The theory of the SI is created by specializing the \isa{Measurement\_System}-type with the 
SI-tag-type and adding new infrastructure. THe SI theory provides the following fundamental 
concepts:
\begin{enumerate}%
\item measuring units and types corresponding to the ISQ base quantities sich
      as \emph{second}, \emph{meter}, \emph{kilogram}, \emph{ampere}, \emph{kelvin}, \emph{mole} and
      \emph{candela} \footnote{together with procedures how to measure a meter, for example, which are
      defined in accompanying standards},
\item a standardized set of symbols for units such as $s$,$m$,$kg$,$A$,$K$,$mol$ and $cd$,
\item a standardized set of symbols of SI prefixes for multiples of SI units, such as 
      $giga$ ($=10^9$), $kilo$ ($=10^3$), $milli$ ($=10^-3$), etc., and a set of
\item \emph{unit equations} and conversion equations such as $J = kg\,m^2/s^2$ or $1 km/h = 1/3.6\,m/s$
\end{enumerate}

As a result, it is possible to express "4500.0 kilogram times meter per second square" which has 
the type 
\isa{{\isasymreal}\ {\isacharbrackleft}M\ \isactrlsup {\isachardot}\ L\ \isactrlsup 
 {\isachardot}\ T\isactrlsup {\isacharminus}\isactrlsup {\isadigit{3}} 
 \isactrlsup {\isachardot}\   T\isactrlsup {\isadigit{1}}, SI{\isacharbrackright}}. 
This type means that the magnitude $4500$ of the dimension 
\isa{M\ \isactrlsup {\isachardot}\ L\ \isactrlsup  {\isachardot}\ T\isactrlsup {\isacharminus}
\isactrlsup {\isadigit{3}} \isactrlsup {\isachardot}\   T\isactrlsup {\isadigit{1}}
} is a quantity intended to be measured in the SI-system, which means that it actually
represents a force measured in Newton. For short, the above expression gets thy type $(\isasymreal)newton$.
In the example, the \emph{magnitude} type of the measurement unit is a real number. 
In general, however, magnitude types can be arbitrary types from the HOL library, so 
for example integer numbers (\isa{int}), integer numbers representable by 32 bits (\isa{int32}), 
IEEE-754 floating-point numbers (\isa{float}), or, 
a vector in the three-dimensional space  \isa{\isasymreal}$^3$. Thus, our type-system allows
to capture both conceptual entities in physics as well as 
implementation issues in concrete physical calculations on a computer.

As mentioned before,  it is a main objective of this work to support the quantity calculus of ISQ 
and the resulting  equations on derived SI entities 
(cf. \url{https://www.quora.com/What-are-examples-of-SI-units}),
both from a type-checking as well as a proof-checking perspective.
Unfortunately, our design objectives are for the case of the Isabelle system somewhat contradictory:
on the one hand, we want a deep integration into the Isabelle/HOL's type-system,
on the other, we have to do normal-form calulations on types.
Isabelle's type system follows the Curry-style paradigm, which rules out the possibility
of direct calculations on type-terms (in contrast to Coq-like systems). However, our semantic
interpretation of ISQ and SI allows for the foundation of the equivalence relation $\_\cong_{Q}\_$
in semantic terms. This paves the way to derived rules that do computations of terms, which
represent type computations indirectly. This principle is the basis for the tactic support we 
developed, which allows for the dimensional type checking of key definitions of the SI 
system such as:

\begin{isamarkuptext}%
\isa{\ \isacommand{theorem}\ metre{\isacharunderscore}definition{\isacharcolon} \newline \ {\isachardoublequoteopen} 
{\isadigit{1}}\ {\isasymodot}\ metre\ {\isasymcong}\isactrlsub Q\ {\isacharparenleft}\isactrlbold c\ \isactrlbold {\isacharslash}\ {\isacharparenleft}{\isadigit{2}}{\isadigit{9}}{\isadigit{9}}{\isadigit{7}}{\isadigit{9}}{\isadigit{2}}{\isadigit{4}}{\isadigit{5}}{\isadigit{8}}\ {\isasymodot}\ {\isasymone}{\isacharparenright}{\isacharparenright}\isactrlbold {\isasymcdot}second{\isachardoublequoteclose}\ {\isachardoublequoteopen}{\isadigit{1}}\ {\isasymodot}\ metre\ {\isasymcong}\isactrlsub Q\ {\isacharparenleft}{\isadigit{9}}{\isadigit{1}}{\isadigit{9}}{\isadigit{2}}{\isadigit{6}}{\isadigit{3}}{\isadigit{1}}{\isadigit{7}}{\isadigit{7}}{\isadigit{0}}\ {\isacharslash}\ {\isadigit{2}}{\isadigit{9}}{\isadigit{9}}{\isadigit{7}}{\isadigit{9}}{\isadigit{2}}{\isadigit{4}}{\isadigit{5}}{\isadigit{8}}{\isacharparenright}\ {\isasymodot}\ {\isacharparenleft}\isactrlbold c\ \isactrlbold {\isacharslash}\ {\isasymDelta}v\isactrlsub C\isactrlsub s{\isacharparenright}{\isachardoublequoteclose}\ 
\newline \isacommand{by}\ si{\isacharunderscore}calc{\isacharplus}\ \ \newline \newline
\isacommand{theorem}\ kilogram{\isacharunderscore}definition{\isacharcolon} \newline \ {\isachardoublequoteopen}{\isacharparenleft}{\isacharparenleft}{\isadigit{1}}\ {\isasymodot}\ kilogram{\isacharparenright}{\isacharcolon}{\isacharcolon}{\isacharprime}a\ kilogram{\isacharparenright}\ {\isasymcong}\isactrlsub Q\ {\isacharparenleft}\isactrlbold h\ \isactrlbold {\isacharslash}\ {\isacharparenleft}{\isadigit{6}}{\isachardot}{\isadigit{6}}{\isadigit{2}}{\isadigit{6}}{\isadigit{0}}{\isadigit{7}}{\isadigit{0}}{\isadigit{1}}{\isadigit{5}}\ {\isasymcdot}\ {\isadigit{1}}{\isacharslash}{\isacharparenleft}{\isadigit{1}}{\isadigit{0}}{\isacharcircum}{\isadigit{3}}{\isadigit{4}}{\isacharparenright}\ {\isasymodot}\ {\isasymone}{\isacharparenright}{\isacharparenright}\isactrlbold {\isasymcdot}metre\isactrlsup {\isacharminus}\isactrlsup {\isasymtwo}\isactrlbold {\isasymcdot}second{\isachardoublequoteclose}\ \newline \isacommand{by}\ si{\isacharunderscore}calc\ \ \ }%
\end{isamarkuptext}\isamarkuptrue%


% \subsubsection{Previous Attempts.} The work of \cite{HayesBrendan95} represents to our knowledge a
% first attempt to formalize SI units in Z, thus a similar language of HOL. While our typing
% representation is more rigourous due to the use of type-classes, this works lacks any attempt
% to support formal and automated deduction on Si unit equivalences.
%
% MORE TO COME.

\chapter{Preliminaries}

\section{More on Integers}
\input{session}

% optional bibliography
\bibliographystyle{abbrv}
\bibliography{root}

\end{document}

%%% Local Variables:
%%% mode: latex
%%% TeX-master: t
%%% End:
